\documentclass[a4paper,11pt]{article}

\newcommand{\workingDate}{\textsc{2022 $|$ January}}
\newcommand{\userName}{Jerry Zheng}
\newcommand{\institution}{}
\usepackage{/Users/jerryzheng/Notes/style/researchdiary_png}
 
\lstset{language=Java,
basicstyle=\ttfamily,
keywordstyle=\color{javapurple}\bfseries,
stringstyle=\color{javared},
commentstyle=\color{javagreen},
morecomment=[s][\color{javadocblue}]{/**}{*/},
tabsize=4,
showspaces=false,
showstringspaces=false}

\DeclareMathAlphabet{\mathcal}{OMS}{cmsy}{m}{n}
\SetMathAlphabet{\mathcal}{bold}{OMS}{cmsy}{b}{n}

\pgfplotsset{compat=1.3}
\begin{document}
% \univlogo
\title*{\Large Amazon summert internship 2023}


\section{initial meeting record}

With Kyungjoo. Potential directions to explore:

\begin{itemize}
    \item Follow-up on squeezed cat 
    \item Fluxonion (w/ bosonic qubit?). 
    \begin{itemize}
        \item Large anharmonicity, in the order of GHz.
        \item Coupling between fluxonion and cat. Similar to the project Connor worked on and presented on March meeting.
        \item A bit inclined towards device physics.
        \item Maybe can also use it for GKP.
    \end{itemize}
    \item Dual-rail. 
    \begin{itemize}
        \item In particular, the treatment of loss. What if not successfully detected loss? These seems to be not considered properly in the previous literature looking at loss.
    \end{itemize}
    \item Fault-tolerance
    \begin{itemize}
        \item Magic state, lattice surgery, and others with biased noise qubit.
    \end{itemize}
\end{itemize}

\section{Second meeting}

With Connor, next meeting tentatively May 23rd. 

\begin{itemize}
    \item Better bias-perserving gates (CNOT in particular, for syndrome extraction) for dissipatively stabilized cat 
    \begin{itemize}
        \item Transmon (already worked out by Connor): idea is using $g/f$ levels to have noise bias. 
        \item Fluxonion
        \begin{itemize}
            \item Benefits (compared to transmon): might not be possible to simultaneously have tho. Need to explore different parameter regime.
            \begin{itemize}
                \item Large noise bias 
                \item Strong dispersive coupling (anharmonicity).
            \end{itemize}
            \item Will Oliver group is also using it for GKP states. The idea is that GKP is sensitive to ancilla bit flip. So can use the noise bias.
            \item We can also explore using tunable couplers to turn on/off the coupling.
            \item Downside: probably T1 on the level of ms? Transmon maybe a few 10's of ms. 
        \end{itemize}
        \item Squeezed cat or GKP as control/target
        \begin{itemize}
            \item Can eliminate non-adiabatic errors. 
            \item For example, if mode A, B are used as control and target modes respectively, the CX gate is realized by the longitudinal coupling 
            \begin{eqnarray}
                    (a +a^\dagger)b^\dagger b
            \end{eqnarray}
            which rotates B is opposite directions conditioned on the logical state of A. However, note that cat codes are not eigenstates of $a+a^\dagger \propto x$, so they shift the states. These shifts are counteracted by dissipation, but still creates non-adiabatic error. For SC or GKP code, they are (approximate) eigenstates of $p$. Therefore, it should not suffer from the non-adiabatic error. Here, there might be some tradeoff between the squeezing of the SC/GKP (larger squeezing larger photon number, larger decoherence introduced) and the reduction in the non-adiabatic error.
            \begin{itemize}
                \item Connor mentioned that this is also the reason why SC is better against non-adiabatic error. Need to check?
            \end{itemize}
        \end{itemize}
    \end{itemize}
    \item Erasure error. Inspired by the Yale paper
    \begin{itemize}
        \item Pervious models of the erasure error is not very realistic. For example, they doesn't take into account the time needed to detect the erasure error. Also, the detection of erasure error is not perfect. What is the effect of this?
        \item How to decode erasure errors. 
        \item Amazon have paper on the dual rail of transmons?
    \end{itemize}
    \item QLDPC code.
    \begin{itemize}
        \item Is there a way to reduce the requirements on long range connectivity based on noise-biased qubits and goodness? For example, we can compromise the goodness in one type of error for the other type of error to be good. For biased-noise qubits, people start with repetition codes. Can we start from classical LDPC codes? Surface code is an example of totally local codes, can we trade the locality for some goodness?
    \end{itemize}
\end{itemize}



\pagebreak


\section{Literature review}

\subsection{Improve performance of repetition cat codes through SC or GKP ancilla}

\subsubsection*{High-performance repetition cat code using fast noisy operations \cite{regent_high-performance_2023}}

Core ideas
\begin{itemize}
    \item Leakage can induce correlated error and increased X error. The previous model simplified that part.
    \begin{itemize}
        \item Solution: add refreshing step (idling)
    \end{itemize}
    \item Ancilla errors have less effect on threshold and resource overhead. It turns out that the non-adiabatic error is only on the ancilla.
    \begin{itemize}
        \item Instead of having the CNOT gate time set to be the optimal gate time (which could be long for small $\kappa_1/\kappa_2$), can fix the gate time to $1/\kappa_2$. This do not give the optimal overall gate error, but gives a better subthreshold scaling (exponent) at the cost of some slight increase in threshold.
        \item For syndrome measurement, can even advance to setting different $\kappa_{1,2}$ values for the ancilla and data qubit since it is the ratio that is hard to decrease.
    \end{itemize}
\end{itemize}

\subsubsection*{Designing High-Fidelity Gates for Dissipative Cat Qubits \cite{gautier_designing_2023}}


The one most relevant to SC and GKP is its observation that the source of non-adiabatic error is the displacement $a+a^\dagger\propto x$. Such an operator acting on the coherent states will induce a position-dependent phase, i.e. a logical Z. In the paper they proposed to engineer a kind of Hamiltonian that is insensitive to the position but still achieve the gate required. 

\tbd{Therefore, if we can have a state that is the eigenstate of the position x we can get rid of the non-adiabatic error on the ancilla?}


\bibliographystyle{unsrt}
\bibliography{/Users/jerryzheng/Zotero/better-bibtex/zotero_all}

\end{document}